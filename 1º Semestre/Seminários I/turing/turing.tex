\documentclass[a4paper,12pt]{article}

\usepackage[utf8]{inputenc}
\usepackage{graphicx}

\title{Alan Turing: Life and Work}
\author{Rodrigo Caetano Rocha}
\date{\today}

\begin{document}
\maketitle

\section{Introduction}

Alan Mathison Turing (23 June 1912 – 7 June 1954) was a British pioneering computer scientist, mathematician, logician, cryptanalyst and theoretical biologist. He was highly influential in the development of computer science, providing a formalisation of the concepts of algorithm and computation with the Turing machine, which can be considered a model of a general purpose computer. Turing is widely considered to be the father of theoretical computer science and artificial intelligence.

During the Second World War, Turing worked for the Government Code and Cypher School (GC\&CS) at Bletchley Park, Britain's codebreaking centre. For a time he led Hut 8, the section responsible for German naval cryptanalysis. He devised a number of techniques for breaking German ciphers, including improvements to the pre-war Polish bombe method and an electromechanical machine that could find settings for the Enigma machine. Turing played a pivotal role in cracking intercepted coded messages that enabled the Allies to defeat the Nazis in many crucial engagements, including the Battle of the Atlantic; it has been estimated that this work shortened the war in Europe by as many as two to four years.

After the war, he worked at the National Physical Laboratory, where he designed the ACE, among the first designs for a stored-program computer. In 1948 Turing joined Max Newman's Computing Laboratory at the University of Manchester, where he helped develop the Manchester computers and became interested in mathematical biology. He wrote a paper on the chemical basis of morphogenesis, and predicted oscillating chemical reactions such as the Belousov–Zhabotinsky reaction, first observed in the 1960s.

Turing was prosecuted in 1952 for homosexual acts, when such behaviour was still a criminal act in the UK. He accepted treatment with oestrogen injections (chemical castration) as an alternative to prison. Turing died in 1954, 16 days before his 42nd birthday, from cyanide poisoning. An inquest determined his death as suicide, but it has been noted that the known evidence is equally consistent with accidental poisoning. In 2009, following an Internet campaign, British Prime Minister Gordon Brown made an official public apology on behalf of the British government for "the appalling way he was treated". Queen Elizabeth II granted him a posthumous pardon in 2013.

\section{Work on computability}

After Sherborne, Turing studied as an undergraduate from 1931 to 1934 at King's College, Cambridge, whence he gained first-class honours in mathematics. In 1935, at the young age of 22, he was elected a fellow at King's on the strength of a dissertation in which he proved the central limit theorem, despite the fact that he had failed to find out that it had already been proven in 1922 by Jarl Waldemar Lindeberg.

In 1928, German mathematician David Hilbert had called attention to the Entscheidungsproblem (decision problem). In his paper "On Computable Numbers, with an Application to the Entscheidungsproblem" (submitted on 28 May 1936 and delivered 12 November), Turing reformulated Kurt Gödel's 1931 results on the limits of proof and computation, replacing Gödel's universal arithmetic-based formal language with the formal and simple hypothetical devices that became known as Turing machines. He proved that some such machine would be capable of performing any conceivable mathematical computation if it were representable as an algorithm. He went on to prove that there was no solution to the Entscheidungsproblem by first showing that the halting problem for Turing machines is undecidable: in general, it is not possible to decide algorithmically whether a given Turing machine will ever halt.

King's College, Cambridge, where Turing was a student in 1931 and became a Fellow in 1935. The computer room is named after him.
Although Turing's proof \cite{turing36} was published shortly after Alonzo Church's equivalent proof using his lambda calculus, Turing's approach is considerably more accessible and intuitive than Church's. It was also novel in its notion of a 'Universal Machine' (now known as a universal Turing machine), with the idea that such a machine could perform the tasks of any other computation machine, or in other words, it is provably capable of computing anything that is computable. John von Neumann acknowledged that the central concept of the modern computer was due to this paper. To this day, Turing machines are a central object of study in theory of computation.

From September 1936 to July 1938, Turing spent most of his time studying under Church at Princeton University. In addition to his purely mathematical work, he studied cryptology and also built three of four stages of an electro-mechanical binary multiplier. In June 1938, he obtained his PhD from Princeton; his dissertation, Systems of Logic Based on Ordinals, introduced the concept of ordinal logic and the notion of relative computing, where Turing machines are augmented with so-called oracles, allowing a study of problems that cannot be solved by a Turing machine.

When Turing returned to Cambridge, he attended lectures given in 1939 by Ludwig Wittgenstein about the foundations of mathematics. Remarkably, the lectures have been reconstructed verbatim, including interjections from Turing and other students, from students' notes. Turing and Wittgenstein argued and disagreed, with Turing defending formalism and Wittgenstein propounding his view that mathematics does not discover any absolute truths but rather invents them.

\cite{5966527,Yang:2010:GCM:1809028.1806606}

\bibliographystyle{plain}
\bibliography{ref}

\end{document}
